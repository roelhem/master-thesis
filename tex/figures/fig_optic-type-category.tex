\documentclass{standalone}

\begin{document}

\begin{figure}

  \begin{center}
    \begin{tikzpicture}[
        xscale=2.4,
        yscale=1.5,
        natto/.style={->},
        dirty/.style={color=black!30}
      ]
  
      \node[] (Kaleidoscope) at (2,3) {$\Kaleidoscope$};
  
      \node[dirty] (Fold) at (-1,6) {$\Fold$};
      \node[] (Setter) at (-0,6) {$\Setter$};
  
      \node[dirty] (Fold1) at (-2,5) {$\FoldOne$};
      \node[dirty] (AffineFold) at (-1,5) {$\AffineFold$};
      \node[] (Traversal) at (0,5) {$\Traversal$};
  
      \node[dirty] (Getter) at (-2,4) {$\Getter$};
      \node[] (Traversal1) at (-1,4) {$\TraversalOne$};
      \node[] (Affine) at (0,4) {$\Affine$};
      \node[dirty] (Review) at (2,4) {$\CReview$};
  
      \node[] (Glass) at (1,4) {$\Glass$};
  
      \node[] (Lens) at (-1,3) {$\Lens$};
      \node[] (Prism) at (0,3) {$\Prism$};
      \node[] (Grate) at (1,3) {$\Grate$};
  
      \node[] (AlgLens) at (-1,2) {$\AlgLens_\Phi$};

      \node[] (CoAlgPrism) at (0,2) {$\CoAlgPrism_\Theta$};

      \node[] (Iso) at (0,1) {$\Iso$};
      \node[] (Eq) at (0,0) {$\Eq$};
  
      \draw[natto] (Eq) -- (Iso);
      \draw[natto] (Iso) -- (AlgLens);
      \draw[natto] (AlgLens) -- (Lens);
      \draw[natto] (Iso) -- (Grate);
      \draw[natto] (Iso) -- (Kaleidoscope);
      \draw[natto] (Iso) -- (CoAlgPrism);
      \draw[natto] (CoAlgPrism) -- (Prism);
      \draw[natto, dirty] (Lens) -- (Getter);
      \draw[natto] (Lens) -- (Traversal1);
      \draw[natto] (Lens) -- (Affine);
      \draw[natto] (Lens) -- (Glass);
      \draw[natto, dirty] (Grate) -- (Review);
      \draw[natto, dirty] (Prism) -- (Review);
      \draw[natto] (Prism) -- (Affine);
  
      \draw[natto] (Grate) -- (Glass);
  
      \draw[natto, dirty] (Getter) -- (Fold1);
      \draw[natto, dirty] (Getter) -- (AffineFold);
      \draw[natto, dirty] (Traversal1) -- (Fold1);
      \draw[natto] (Traversal1) -- (Traversal);
      \draw[natto] (Affine) -- (Traversal);
      \draw[natto, dirty] (Affine) -- (AffineFold);
      \draw[natto, dirty] (Fold1) -- (Fold);
      \draw[natto, dirty] (AffineFold) -- (Fold);
      \draw[natto, dirty] (Traversal) -- (Fold);
      \draw[natto] (Traversal) -- (Setter);
      \draw[natto] (Glass) -- (Setter);
  
  
      \draw[natto] (Kaleidoscope) -- (Setter);
      \draw[natto, dirty] (Kaleidoscope) -- (Review);
      %\draw[natto] (Fold) -- (Optic);
      %\draw[natto] (Review) -- (Optic);
  
    \end{tikzpicture}
  \end{center}

  \caption{The start of the category of optic types. An arrow can be read as a natural transformation between optic types. The goal is give a better explanation of this diagram and to see if I can make it a proper lattice.}
  \label{fig:optic-type-category}
\end{figure}


\end{document}